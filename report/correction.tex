\section{Correction to the Newton's law}

In this section, we will first recall the expression of the total energy of a test particle of mass $m$ in a central force field
generated by a mass $M$. Then we will derive an equivalent expression in the relativistic case. To do so, we will start
from the geodesic equation of the test particle in the Schwarzschild metric due to the presence of the mass $M$. Then comparing
the two expression we will deduce the corrective term.

\subsection{Central force field in classical mechanics}

Let us consider the motion of a test particle of mass $m$ in central force field, $\textbf{F}(\textbf{r}) = f(r)\textbf{e}_r$
with the origin as its force center. The force derives from a potential energy $V(r)$ which is function of $r$ alone and then the Lagrangian
associated to the problem is $\mathcal{L} = T - V = \frac{1}{2}m v^2 - V(r)$. In polar coordinates, the velocity has for expression
$\mathbf{v}= \dot{r}\mathbf{e}_r + r\dot{\theta} \mathbf{e}_\theta$, then the expressions of the Lagrangian becomes :
%
\begin{equation}
 \mathcal{L} = \frac{1}{2}m\left(\dot{r}^2 + r^2 \dot{\theta}^2\right) - V(r)
\end{equation}
%
Using the Euler-Lagrange equations : 
%
\begin{equation}
 \frac{\mathrm{d}}{\mathrm{d}t}\left( \frac{\partial \mathcal{L}}{\partial \dot{q}_i} \right) = \frac{\partial \mathcal{L}}{\partial q_i}
\end{equation}
%
we obtain :
%
\begin{equation}\label{conservation_1}
  \frac{\mathrm{d}}{\mathrm{d}t}\left( mr^2\dot{\theta}\right) = 0
\end{equation}
%
\begin{equation}\label{conservation_2}
   m\ddot{r} - mr\dot{\theta}^2 = -\frac{\partial V(r)}{\partial r}
\end{equation}
%
The first equation is easy to integrate and we obtain a conserved quantity $mr^2\dot{\theta} = J$.
It is nothing else than the magnitude of the angular momentum vector $ \mathbf{J} = \mathbf{r} \times \mathbf{p}$.
Now injecting this relation in \eqref{conservation_2} and integrating, we obtain a second conserved quantity which is the total energy
of the system :
%
\begin{equation}
 E = \frac{1}{2}m\dot{r}^2 + \frac{1}{2}\frac{J^2}{mr^2} + V(r)
\end{equation}
%
In our particular case, we have $V(r) = - \frac{2GmM}{r}$ and normalizing the expression by the mass $m$ of the test particle, we obtain :
%
\begin{equation}\label{classical_relation}
  e = \frac{1}{2}\dot{r}^2 + \frac{1}{2}\frac{h^2}{r^2} -\frac{2GM}{r}
\end{equation}
%
where $e = E/m$ and $h=J/m$.

\subsection{Relativistic case and corrective term}

First, we have to derive geodesic equation for the test particle. In this objective, we inject expression of the Christoffel symbols
\eqref{christoffel} in \eqref{formula:geodesic}. We find :
%
\begin{subequations}\label{schwarz_geodesic}
  \begin{align}
	\lambda = 0 &: \qquad \ddot{t} + \frac{A'}{A}
	\dot{t}\dot{r}
	=0 \\
	\lambda = 1 &: \qquad \ddot{r} + \frac{B'}{2B}{\dot{r}}^2 + \frac{A'}{2B}{\dot{t}}^2
	- \frac{r}{B} {\dot{\theta}}^2 - \frac{r}{B}\sin^2(\theta) {\dot{\phi}}^2
	=0 \\
	\lambda = 2 &: \qquad \ddot{\theta} + \frac{2}{r}\dot{r}\dot{\theta}
	- \sin(\theta)\cos(\theta){\dot{\phi}}^2
	= 0 \\
	\lambda = 3 &: \qquad \ddot{\phi} + \frac{2}{r}\dot{r}\dot{\phi} + 2\cot(\theta)\dot{\theta}\dot{\phi}
	=0
  \end{align}
\end{subequations}
%
where dot means differentiation with respect to $\tau$ and prime holds for differentiation with respect to $r$.
We now consider in the remaining of our study and without loss of generality (since the field is isotropic) that the movement
takes place in the plan $\theta = \frac{\pi}{2}$. So previous equations \eqref{schwarz_geodesic} become :
%
\begin{subequations}
  \begin{align}
	&\ddot{t} + \frac{A'}{A} \dot{t}\dot{r} =0 \label{machin_1} \\
	&\ddot{r} + \frac{B'}{2B}{\dot{r}}^2 + \frac{A'}{2B}{\dot{t}}^2 - \frac{r}{B}{\dot{\phi}}^2 = 0 \label{machin_2}\\
	&\ddot{\phi} + \frac{2}{r}\dot{r}\dot{\phi} = 0 \label{machin_3}
  \end{align}
\end{subequations}
%
and equation for $\lambda = 2$ is no longer pertinent.
Multiplying equation \eqref{machin_1} by $A$ and equation \eqref{machin_3} by $r^2$ we can rewrite them :
%
\begin{subequations}
\begin{align}
 &\frac{\mathrm{d}}{\mathrm{d}\tau}\left(\dot{t}A(r)\right) = 0\\
 &\frac{\mathrm{d}}{\mathrm{d}\tau}\left(r^2\dot{\phi}\right) = 0
 \end{align}
\end{subequations}
%
so the quantities :
%
\begin{subequations}
 \begin{align}
  &A(r) \dot{t} \equiv \sqrt{2e + 1}\label{conservation_energy}\\
  &r^2\dot{\phi} \equiv h \label{conservation_angular}
  \end{align}
\end{subequations}
%
are conserved along the trajectory. For the moment, $e$ can be taken as an arbitrary constant but in fact we will that it is 
the total energy of the system.
Now by introducing these relations in \eqref{machin_2} and multiplying it by $2B\dot{r}$, we can rewrite it as :
%
\begin{equation}
 \frac{\mathrm{d}}{\mathrm{d}\tau}\left(B{\dot{r}}^2 - \frac{2e+1}{A(r)} + \frac{h^2}{r^2}\right) = 0
\end{equation}
%
Then, we obtain one more conserved quantity : 
%
\begin{equation} \label{machin_4}
  B(r)\dot{r}^2 + \frac{h^2}{r^2} - \frac{2e+1}{A(r)} = k  
\end{equation}
%
But from the expression of the metric we deduce an other relation :
\begin{equation}
 1 = A(r)\dot{t}^2 - B(r) \dot{r}^2 - r^2\dot{\phi}^2 
\end{equation}
%
(we recall that $\theta = \pi/2$). After injecting relations \eqref{conservation_energy} and \eqref{conservation_angular}, previous equation becomes :
\begin{equation}
 1 = \frac{2e+1}{A(r)} - B(r) \dot{r}^2 - \frac{h^2}{r^2} = -k
\end{equation}
%
Then we see that we have $k=-1$.
Multiplying by $A$ and replacing by its expression in equation \eqref{machin_4}, we can put it in the form:
%
\begin{equation}
 e = \frac{\dot{r}^2}{2} - \frac{GM}{r} + \frac{h^2}{2r^2} - \frac{GMh^2}{r^3}
\end{equation}
%
which is the same as equation \eqref{classical_relation} excepted for the term proportional to $\frac{1}{r^3}$ which is of course 
the excepted corrective term. From this and 
reintroducing the mass $m$ of the test particle, we deduce the expression of the corrected potential energy :
%
\begin{equation}
 V_{\mathrm{corr.}}(r) = -\frac{GmM}{r} - \frac{GMJ^2}{mr^3}
\end{equation}
%
and then the expression of the corresponding force :
%
\begin{equation}
	 \mathbf{F}_{\mathrm{corr.}}(r) = - \frac{\mathrm{d}}{\mathrm{d}r} \left(V(r)\right)\mathbf{e}_r = \left[-\frac{GmM}{r^2} - \frac{3GMJ^2}{mr^4}\right] \mathbf{e}_r
\end{equation}
%
So the classical force is corrected by an attractive term proportional to $\frac{1}{r^4}$. As we will see in the next section, this term is responsible for
the precession of the perihelion of Mercury and it is because it is non negligible for short distances that this phenomena was visible only in the case of Mercury.
Now we know that it affects all the planets in solar system, included the Earth.