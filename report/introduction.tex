\section{Introduction}

About three centuries ago, Newton formulated the Three Laws of Motion in his famous
\textit{Philosophiae naturalis principia mathematica}, and laid the foundations of what we
call now Newtonian mechanics or, since Einstein, classical mechanics. In this theory, the gravitation
is described as a force responsible for interactions between massive objects.
However, a few physical phenomena remained elusive and could not enter in the frame of Newton
theory, such as the orbit precession of Mercury. At the beginning of the twentieth century,
a new theory of gravitation came up, mainly established by Einstein, which supplanted Newton’s
law and which is still the most accurate theory of gravitation so far.
In this theory, the concept of force is totally absent and gravitation is understood as a modification
of the space-time geometry under the influence of energy and momentum coming from matter or radiation.
This modification is given by the set of ten non-linear equations known as Einstein field equations.
The Schwarzschild metric is the simpler analytical solution of these equations and describes the space-time geometry
near an homogeneous, static and isotropic spherical body as a star or a planet.
Nevertheless, we can legitimately wonder how Newton’s theory would be corrected by General Relativity.
The main goal of this project is to consider the Schwarzschild metric to figure out the correction provided by General
Relativity to Newton’s law, to give a physical interpretation of it and whenever possible,
to apply it to a concrete case like the perihelion precession of Mercury.\newline\newline
%
\textit{In our report, we use Einstein summation convention consisting of summing
on repeated indices which appears once as a subscript and once as a superscript.}\newline
\textit{Latin indices run over three spatial coordinates labels, while greek indices run
over four coordinates.}\newline
$\eta_{\mu\nu}$\textit{ denotes the Minkowski metric and we use the signature} $(+,-,-,-)$.\newline
\textit{Finally, we use natural units where }$c=1$.