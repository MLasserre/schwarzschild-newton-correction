\section{Schwarzschild metric}

Here we are going to study the Schwarzschild metric, which is an analytical solution to
the Einstein field equations in the static isotropic case. We will need some symmetric views and
tensorial calculus to arrive at its expression.

\subsection{The general static isotropic metric}

We are going to find the most general metric tensor that can represent a static isotropic
gravitational field. The symmetry arguments are essential. 

We consider the invariant proper time in spherical coordinates :
%
\begin{equation}\label{minkowski_spherical}
	\mathrm{d}\tau^2 = \mathrm{d}t^2-\mathrm{d}r^2-r^2(\mathrm{d}\theta ^2+\sin^2\theta
	\mathrm{d}\phi^2)
\end{equation}
%
The term $\mathrm{d}\theta ^2+\sin^2\theta \mathrm{d}\phi^2$ has symmetric and isotropic
characteristics.
We can generalize it to an equation for any spherically symmetric metric :
%
\begin{equation}
	\mathrm{d}\tau^2=A\mathrm{d}t^2-B\mathrm{d}t\mathrm{d}r-Cr^2-D(\mathrm{d}\theta ^2
	+\sin^2\theta \mathrm{d}\phi^2)
\end{equation}
%
where $A$, $B$, $C$ and $D$ are functions of the coordinates.
For symmetric and isotropic reasons, these functions cannot depend on $\theta$
or $\phi$. And since we expect the symmetry under the transformation $\phi$ $\rightarrow$
$-\phi$ and $\theta$ $\rightarrow$ $\pi-\theta$, there are no cross terms such
$\mathrm{d}r\mathrm{d}\theta$ or $\mathrm{d}\phi \mathrm{d}t$.

We define a new radial coordinate $r'$ such that $\left(r'\right)^2=D$ :
%
\begin{equation}
	\mathrm{d}\tau^2=A'(t,r')\mathrm{d}t^2-B'(t,r')\mathrm{d}t\mathrm{d}r'-C'(t,r')r'^2-\left(r'\right)^2
	\left(\mathrm{d}\theta ^2+\sin^2\theta \mathrm{d}\phi^2\right)
\end{equation}
%
and we can finally transform the time coordinate :
%
\begin{equation}
	\mathrm{d}t'=f\mathrm{d}t+g\mathrm{d}r'
\end{equation}
%
with f and g arbitrary functions\footnote{We are free to measure time as we want.}. This can allow us to eliminate the cross term
$\mathrm{d}r'\mathrm{d}t$ \footnote{This can be seen as diagonalization of the matrix $g_{\mu\nu}$.}.
At last, we apply the static condition and know that $A'$ and $B'$ do not depend on $t'$.
Finally, by leaving quotes aside, we have :
\begin{equation}\label{symmetry}
	\mathrm{d}\tau^2=A\left(r\right)\mathrm{d}t^2-B\left(r\right)
	\mathrm{d}r^2-r^2\mathrm{d}\theta ^2-r^2\sin^2\theta \mathrm{d}\phi^2
\end{equation}

This is the standard form of an static isotropic metric.

\subsection{Derivation of the Schwarzschild metric}

To get the Schwarzschild metric, that is by determining the functions $A$ and $B$, we apply the Einstein field equations \eqref{Einstein} to the general
static isotropic metric \eqref{symmetry}.

By definition, Schwarzschild metric describes structure of space-time outside a spherical, neutral and non-rotating mass in an empty space. Then, as we said previously, the
stress-energy tensor $T_{\mu\nu}$ is null and according to \eqref{scalar_curvature_stress_energy}, so as the scalar curvature. In this case, Einstein
field equations \eqref{Einstein} reduces to $R_{\mu\nu}=0$.

In order to find components of the Ricci tensor, we need to calculate the Christoffel symbols
$\Gamma^{\sigma}_{\mu\nu}$ from the metric $g_{\mu\nu}$. So first we need to identify the
metric components from equation \eqref{symmetry} using \eqref{invariant}. For the components
of $g_{\mu\nu}$, we label $x^0=t$, $x^1=r$, $x^2=\theta$, $x^3=\phi$ then :
%
\begin{equation}
g_{\mu\nu}=\left(
	\begin{array}{cccc}
	g_{00} & g_{01} & g_{02} & g_{03}\\
	g_{10} & g_{11} & g_{12} & g_{13}\\
	g_{20} & g_{21} & g_{22} & g_{23}\\
	g_{30} & g_{31} & g_{32} & g_{33}\\
	\end{array}
\right)=\left(
	\begin{array}{cccc}
	g_{tt} & g_{tr} & g_{t\theta} & g_{t\phi}\\
	g_{rt} & g_{rr} & g_{r\theta} & g_{r\phi}\\
	g_{\theta t} & g_{\theta r} & g_{\theta\theta} & g_{\theta\phi}\\
	g_{\phi t} & g_{\phi r} & g_{\phi\theta} & g_{\phi\phi}\\
		\end{array}
\right)
\end{equation}

Consequently, non-vanishing elements of the metric
$g_{\mu\nu}$ and its inverse $g^{\mu\nu}$ are :
%
\begin{equation}
	g_{00}=A\left(r\right),	\qquad g^{00}=1/A\left(r\right)
\end{equation}
%
\begin{equation}
	g_{11}=-B\left(r\right),\qquad g^{11}=-1/B\left(r\right)
\end{equation}
%
\begin{equation}
	g_{22}=-r^2,\qquad g^{22}=-1/r^2
\end{equation}
%
\begin{equation}
	g_{33}=-r^2\sin^2\theta,\qquad g^{33}=-1/\left(r^2\sin^2\theta\right)
\end{equation}

Now we can apply the formula \eqref{formula:christoffel} and calculate the Christoffel
symbols $\Gamma^{\sigma}_{\mu\nu}$. We recall that Christoffel symbols are symmetric by
interchanging $\mu$ and $\nu$. The non-vanishing components are :
%
\begin{equation}\label{christoffel}
	\Gamma^{0}_{01}=A'/\left(2A\right),\qquad \Gamma^{1}_{00}=A'/\left(2B\right),
	\qquad \Gamma^{1}_{11}=B'/\left(2B\right)
\end{equation}
%
\begin{equation}
	\Gamma^{1}_{22}=-r/B,\qquad \Gamma^{1}_{33}=-\left(r\sin^2\theta\right) /B,
	\qquad \Gamma^{2}_{12}=1/r
\end{equation}
%
\begin{equation}
	\Gamma^{2}_{33}=-\sin\theta \cos\theta ,\qquad \Gamma^{3}_{13}=1/r,
	\qquad \Gamma^{3}_{23}=\cot\theta
\end{equation}

Now we substitute the expression of $\Gamma^{\sigma}_{\mu\nu}$ into $R_{\mu\nu}$. Only the diagonal
components are non-vanishing :
%
\begin{equation}
	R_{00}=-\frac{A''}{2B}+\frac{A'}{4B}\left(\frac{A'}{A}+\frac{B'}{B}\right)-\frac{A'}{rB}
\end{equation}
%
\begin{equation}
	R_{11}=\frac{A''}{2A}-\frac{A'}{4A}\left(\frac{A'}{A}+\frac{B'}{B}\right)-\frac{B'}{rB}
\end{equation}
%
\begin{equation}\label{R22}
	R_{22}=\frac{1}{B}-1+\frac{r}{2B}\left(\frac{A'}{A}-\frac{B'}{B}\right)
\end{equation}
%
\begin{equation}
	R_{33}=R_{22}\sin^2\theta
\end{equation}
%
(A prime means differentiation with respect to $r$).
The fourth equation is merely the consequence of the first three. From these equations, we can establish the relation :
%
\begin{equation}
	A'B+AB'=0
\end{equation}
%
In other words, $AB$=constant. By imposing the boundary condition that when $r\rightarrow \infty$ the metric tensor must approach
the Minkowski metric \eqref{minkowski_spherical}, we have $A(r)\rightarrow 1$ and $B(r) \rightarrow 1$. Then the constant of integration is equal to $1$.
Now, substituting this relation into \eqref{R22}, we get :
%
\begin{equation}
	\frac{\mathrm{d}\left(rA\right)}{\mathrm{d}r}= 1
\end{equation}

By integrating this equation, we can find easily :
%
\begin{equation}\label{function_A_and_B}
	A\left(r\right) = 1 + \frac{k}{r} \quad \mathrm{and}
	\quad B\left(r\right) = \left(1+\frac{k}{r}\right)^{-1}
\end{equation}

where $k$ is an integration constant.

Far from the mass, we can place us in weak field approximation, that is $g_{\mu\nu} = \eta_{\mu\nu} + h_{\mu\nu}$ with $\left| h_{\mu\nu}\right| \ll 1$. In
this limit and for any metric, we have the relation\footnote{See \cite{weinberg1972gravitation}.} $g_{00} = 1 + 2\phi$ where $\phi = -\frac{GM}{r}$ is
the classical gravitational field.
Considering this limit we find that $k=-2GM$ so injecting this expression in \eqref{function_A_and_B}, we find that
$A(r) = 1 - \frac{2GM}{r}$ and $B(r) = \left( 1-\frac{2GM}{r}\right)^{-1}$. Finally we obtain the Schwarzschild metric :
%
\begin{equation}\label{schwa}
	\mathrm{d}\tau^2=\left(1-\frac{2GM}{r}\right)\mathrm{d}t^2
	-\left(1-\frac{2GM}{r}\right)^{-1}\mathrm{d}r^2-r^2\left(\mathrm{d}\theta ^2
	+\sin^2\theta \mathrm{d}\phi^2\right)
\end{equation}
%
As the Schwarzschild metric describes static isotropic
gravitational field, it is a good candidate to describe the movement of the planets. In the next part so, by
applying this metric we will search for a correction term in Newton's law brought by General Relativity.